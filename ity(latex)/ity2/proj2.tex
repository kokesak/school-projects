\documentclass[a4paper,11pt]{article}
\usepackage[left=1.5cm,text={18cm, 25cm},top=2.5cm]{geometry}
\usepackage[czech]{babel}
\usepackage[IL2]{fontenc}
\usepackage{times}
\usepackage[utf8]{inputenc}
\usepackage[unicode]{hyperref}
\usepackage{changepage}
\usepackage{multicol}
\usepackage{amsthm, amssymb, amsmath}
\theoremstyle{definition}
\newtheorem{definition}{Definice}
\newtheorem{sentence}{Věta}
%\setlength{\parindent}{0.4em}

\begin{document}

\begin{center}
\Huge
\textsc{Fakulta informačních technologií}\\
\textsc{Vysoké učení technické v Brně}\\
\vspace{\stretch{0.382}}
\LARGE
Typografie a publikování\,--\,2. projekt\\
Sazba dokumentů a matematických výrazů\\
\vspace{\stretch{0.618}}
\end{center}
%\newpage
{\LARGE 2020 \hfill Martin Litwora (xlitwo00)}
\thispagestyle{empty}
\newpage

\setcounter{page}{1}
\twocolumn
%\begin{multicols}{2}
\section*{Úvod}
V této úloze si vyzkoušíme sazbu titulní strany, matematických vzorců, prostředí a dalších textových struktur obvyklých pro technicky zaměřené texty (například rovnice (\ref{eq2}) nebo Definice \ref{def2} na straně \pageref{def1}). Pro vytvoření těchto odkazů používáme příkazy \verb|\label|, \verb|\ref| a \verb|\pageref|.

Na titulní straně je využito sázení nadpisu podle optického středu s využitím zlatého řezu. Tento postup byl probírán na přednášce. Dále je použito odřádkování se zadanou relativní velikostí 0.4em a 0.3em.

\section{Matematický text}
Nejprve se podíváme na sázení matematických symbolů a~výrazů v plynulém textu včetně sazby definic a vět~s~využitím balíku \texttt{amsthm}. Rovněž použijeme poznámku pod čarou s použitím příkazu \verb|\footnote|. Někdy je vhodné použít konstrukci \verb|${}$| nebo \verb|\mbox{}| která říká, že (matematický) text nemá být zalomen. V následující definici je nastavena mezera mezi jednotlivými položkami \verb|\item| na 0.05em.

\begin{definition} \label{def1}
Turingův stroj \emph{(TS) je definován jako šestice tvaru $M = (Q, \Sigma, \Gamma, \delta, q_0,q_F$), kde:}
\end{definition} 
\begin{itemize}
    \setlength\itemsep{0.05em}
    \item \emph{$Q$ je konečná množina} vnitřních (řídicích) stavů,
    \item \emph{$\Sigma$ je konečná množina symbolů nazývaná} vstupní abeceda, $\Delta \notin \Sigma$,
    \item \emph{$\Gamma$ je konečná množina symbolů, $\Sigma \subset \Gamma$, $\Delta \in~\Gamma$, nazývaná} pásková abeceda,
    \item $\delta$ : \emph{$\mbox{${(Q\backslash \{q_F\})\!\times\!\Gamma\!\rightarrow\!Q\!\times\!(\Gamma\!\cup\!\{L,R\}),}$\,kde\,${L,R\!\notin\!\Gamma}$}$, je parciální} přechodová funkce, \emph{a}
    \item ${q_0} \in Q$ \emph{je} počáteční stav a ${q_F \in Q}$ \emph{je} koncový stav.
\end{itemize}

Symbol $\Delta$ značí tzv. \emph{blank} (prázdný symbol), který se vyskytuje na místech pásky, která nebyla ještě použita.

\emph{Konfigurace} pásky se skládá z nekonečného řetězce, který reprezentuje obsah pásky a pozice hlavy na tomto řetězci. Jedná se o prvek množiny $\{\gamma \Delta^\omega \mid \gamma \in \Gamma^*\}\times \mathbb{N}\footnote{Pro libovolnou abecedu $\Sigma$ je $\Sigma^\omega$ množina všech \emph{nekonečných} řetězců nad $\Sigma$, tj. nekonečných posloupností symbolů ze $\Sigma$.}.$
\emph{Konfiguraci} pásky obvykle zapisujeme jako 
${\Delta xyz\underline{z}x\Delta}$...
(podtržení značí pozici hlavy). \emph{Konfigurace} stroje je pak dána stavem řízení a konfigurací pásky. Formálně se jedná o prvek množiny ${Q\times\{\gamma \Delta^\omega \mid \gamma \in \Gamma^*\}\times \mathbb{N}}$.

\subsection{Podsekce obsahující větu a odkaz}
\begin{definition}\label{def2}
Řetězec $w$ nad abecedou $\Sigma$ je přijat TS $M$ \emph{jestliže $M$ při aktivaci z počáteční konfigurace pásky \\
$\underline{\Delta} w \Delta...$ a počátečního stavu $q_0$ zastaví přechodem do koncového stavu $q_F$, tj. $(q_0, \Delta w \Delta^\omega,0) \underset{M}{\overset{*}{\vdash}} (q_F,\gamma,n)$ pro nějaké $\gamma \in \Gamma^*$ a $n \in \mathbb{N}$.}

\emph{Množinu ${L(M)\!=\!\{w \mid w}$ je přijat TS ${M\}\subseteq \Sigma^*}$ nazýváme} jazyk přijímaný TS $M$.
\end{definition}

Nyní si vyzkoušíme sazbu vět a důkazů opět s použitím balíku \texttt{amsthm}.

\begin{sentence}
\emph{Třída jazyků, které jsou přijímány TS, odpovídá} rekurzivně vyčíslitelným jazykům.
\end{sentence}

\begin{proof}
V důkaze vyjdeme z Definice \ref{def1} a \ref{def2}.
\end{proof}

\section{Rovnice}\label{eq2}
Složitější matematické formulace sázíme mimo plynulý text. Lze umístit několik výrazů na jeden řádek, ale pak je třeba tyto vhodně oddělit, například příkazem \verb|\quad|.
\bigskip
$$ \sqrt[i]{x^3_i} \quad \textnormal{ kde $x_i$ je $i$-té sudé číslo} \quad y^{2 \cdot y_i}_i \neq y_i^{y_i^{y_i}} $$

V rovnici (\ref{eq1}) jsou využity tři typy závorek s různou explicitně definovanou velikostí.
\bigskip
\begin{eqnarray} \label{eq1}
x & = & \bigg\{ \Big( \big[ a + b \big] * c \Big)^d \oplus 1 \bigg\} \\
y & = & \lim_{x \to \infty} \frac{\sin^2{x} + \cos^2{x}}{\frac{1}{\log_{10}x}}
\end{eqnarray}

V této větě vidíme, jak vypadá implicitní vysázení limity $\lim_{n \to \infty}f(n)$ v normálním odstavci textu. Podobně je to i s dalšími symboly jako $\sum_{i=1}^n 2^i$ či $\bigcap_{A \in \mathcal{B}} A$. V~případě vzorců $\lim\limits_{n \to \infty}f(n)$ a $\sum\limits_{i=1}^n 2^i$ jsme si vynutili méně úspornou sazbu příkazem \verb|\limits|.

\section{Matice}
Pro sázení matic se velmi často používá prostředí \texttt{array} a závorky (\verb|\left|,\verb|\right|).
$$ \left(
\begin{array}{ccc} 
a + b & \widehat{\xi + \omega} & \hat{\pi} \\
\vec{\mathbf{a}} & \overleftrightarrow{AC} & \beta
\end{array} 
\right) = 1 \Longleftrightarrow \mathbb{Q} = \mathcal{R}$$
Prostředí \texttt{array} lze úspěšně využít i jinde.
$$  \binom{n}{k} = 
    \left\{
    \begin{array}{cl}
        0 & \text{pro}\ k < 0 \ \text{nebo}\ k > n  \\
        \frac{n!}{k!(n-k)!} &  \text{pro}\ 0 \leq k \leq n.
    \end{array}
    \right . $$
%\enlargethispage{10pt}
%\end{multicols}
\end{document}
