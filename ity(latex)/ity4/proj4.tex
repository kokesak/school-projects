\documentclass[a4paper,11pt]{article}
\usepackage[left=2cm,text={17cm, 24cm},top=3cm]{geometry}
\usepackage[czech]{babel}
\usepackage{times}
\usepackage[utf8]{inputenc}
\usepackage[unicode]{hyperref}
\usepackage{changepage}

\begin{document}

\begin{titlepage}
\begin{center}
\Huge
\textsc{Fakulta informačních technologií}\\
\textsc{Vysoké učení technické v Brně}\\
\vspace{\stretch{0.382}}
\huge
Typografie a publikování\,--\,4. projekt\\
\Huge
Bibliografické citace\\
\vspace{\stretch{0.618}}
\end{center}
{\LARGE \today \hfill Martin Litwora}
\end{titlepage}


\section{Co znamená typografie?}
Původně typografie pochází z kombinací řeckých slov "typos" (forma) a "graphia" (psaní). Dohromady tedy typografie znamená psaní v souladu s formou. Velký rozvoj typografie zaznamenala po vynálezu knihtiksu v roce 1448. Vznikly tehdy dvě formy latinského písma, a to písma humanistická a novogotická. Další informace zde \cite{historie}.

Dneska je typografie některými považovaná za umění, viz \cite{clanek_konference1}. Například Martin Solomon (1995) vyjádřil svůj názor následovně: "Typografie je umění mechanicky vytvářet písmena, čísla,  symboly, a tvary v souladu se základními principy a vlastnostmi designu."\cite{Solomon}.

\section{Kreativní typografie}
Text se používá primárně pro předávání informací, ovšem změnou barvy, fontu, přidání grafických elementů můžeme z textu vytvořit obrázek, který v nás může probouzet různé pocity. Dobrým příkladem je socha LOVE, která je umístěna na Manhattanu. Podrobněji zde \cite{creative_typography}.

Použití barev může čtenáři napomoct navázat vztah s objektem, o kterém se mluví a začít s tím více sympatizovat. Například, studie ukázala, že červená barva zvyšuje tlukot srdce, což vede k většímu vzrušení. Viz~zde~\cite{Donev}. 

\section{Velikost písma u dětí}
Na dětech ve věku od 7 od 9 let byly provedeny testy, které měly zjistit jak se velikost fontu projeví na rychlosti čtení. Použit byl font Arial o velikosti 26pt a 22pt, výsledkem bylo, že větší písmo děti zvládli přečíst o 9 \% rychleji. Více viz \cite{kids}.

Obdobná studie, kterou provedli pánové M. Tinker a  D. Paterson se zaobírala použitím kapitálek pro nebo obyčejného malého písma. Obyčejný text připadal čtenářům více přívětiví u delších textů a četl se rychleji. Podrobněji zde \cite{Guthrie}.

\section{Typografické jednotky}
Existuje několik typografických měrných soustav, jedna z nejznámějších je anglosaská (monotypová) měrná soustava:
\begin{itemize}
    \item 1" (inch, palec) = 72,27 points (= 2,54 cm)
    \item 1 point = 0,351 mm
    \item 1 pica [pajka] (stupeň písma) = 12 points = 4,217 mm
\end{itemize}
Převzato z \cite{jednotky}.

\section{Fonty}
Fonty rozdělujeme do několika kategorií. Patková písma (serif) používají patky a neunavují tolik čtenářovo oko. Používají se pro rozsáhlejší texty. Nejnzámějším zástupcem je Times New Roman (který se však používá spíše pro novinové články).
Bezpatková písma (sans-serif) se používají jen pro kratší texty, v rozsahu jen jednoho až dvou odstavců. Z běžných fontů jde o Arial.
Neproporcionální písma mají stejnou šířku a tak se používají pro sazbu zdrojových kódů.
Dekorativní fonty mají za cíl přitáhnout pozornost. Nehodí se však na pohodlnou četbu. Pro více informací zde \cite{fonty}.

\section{Unicode}
Cílem Unicode je přiřadit unikátní ID každému znaku, lingvistickému symbolu, nebo ideogramu ze všech světových jazyků, at uz mrtvých nebo stále použivaných. Počet těchto unikátních ID v dnešní době přesahuje 100 000. Zabralo to několik desetiletí, než byl v IT tento mezinárodní standart zaveden. Dále zde \cite{unicode}.

\bibliographystyle{czechiso}
\bibliography{references}
\end{document}
